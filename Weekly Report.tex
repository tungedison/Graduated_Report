\documentclass[12pt,a4paper]{article}  
%\usepackage[vietnam]{babel}
\usepackage[utf8]{vietnam}
%\usepackage{type1cm}
\usepackage{a4wide,amssymb,epsfig,latexsym,multicol,array,hhline,fancyhdr}
\usepackage{lastpage} 
\usepackage[left=2.5cm,right=1.5cm,top=1.5cm,bottom=2.5cm]{geometry}
\usepackage{enumerate}
\usepackage{color}
\usepackage{graphicx}
\usepackage{array}
\usepackage{tabularx}
\usepackage{amsmath}
\usepackage{amssymb}
\usepackage{exscale}
\usepackage{eucal}
\usepackage{multirow}
\usepackage{indentfirst} %Thụt vào đầu dòng khi bắt đầu đoạn văn bản mới.
\usepackage{listings}
\usepackage{multicol}
\usepackage{caption} 
\usepackage{rotating}
\usepackage{graphics}
\usepackage{geometry}
\usepackage{setspace}
\usepackage{epsfig}
\usepackage{subfig}
\usepackage{wrapfig}
\usepackage{hyperref}
%\usepackage{subfigure}
\usepackage{tikz}
%\usepackage{placeins}
\usepackage{pifont}
\usepackage{appendix}
\usepackage{booktabs}
\usetikzlibrary{arrows,snakes,backgrounds}
\newtheorem{theorem}{{\bf Theorem}}
\newtheorem{property}{{\bf Property}}
\newtheorem{proposition}{{\bf Proposition}}
\newtheorem{corollary}[proposition]{{\bf Corollary}}


\newcounter{numproblem}
\newenvironment{problem}{\addtocounter{numproblem}{1}
\noindent{\large \bf Problem \thenumproblem. }}{}
\newcounter{numexercise}
\newenvironment{exercise}{\addtocounter{numexercise}{1}
\noindent{\large \bf Câu \thenumexercise. \\}}{}

%\newcounter{numsolution}
\newenvironment{solution}{
\noindent{\large \bf Lời giải.}\color{blue}}{~~\hfill$\Box$\\}

\newif\ifshortversion
\shortversiontrue 	% hide this line when shortversion==false
\newcommand\version[2]{\ifshortversion #1 \else #2 \fi}
\newcommand{\specialcell}[2][c]{%
	\begin{tabular}[#1]{@{}c@{}}#2\end{tabular}}
%\usepackage{fancyhdr}
\setlength{\headheight}{40pt}
\pagestyle{fancy}
\fancyhead{} % clear all header fields
\fancyhead[L]{
 \begin{tabular}{rl}
    \begin{picture}(25,15)(0,0)
    \put(0,-8){\includegraphics[width=8mm, height=8mm]{LogoBK.jpg}}
    %\put(0,-8){\epsfig{width=10mm,figure=hcmut.eps}}
   \end{picture}&
	%\includegraphics[width=8mm, height=8mm]{hcmut.png} & 
	\begin{tabular}{l}
		{ {\texttt{Trường Đại Học Bách Khoa}}}\\
		{  {\texttt{Khoa Khoa Học \& Kỹ Thuật Máy Tính}}}
	\end{tabular} 	
 \end{tabular}
}
\fancyhead[R]{
	\begin{tabular}{l}
		\tiny \bf \\
		\tiny \bf 
	\end{tabular}  }
\fancyfoot{} % clear all footer fields
%\fancyfoot[L]{\scriptsize \texttt{ Đề tài: Năng lượng sạch}}
\fancyfoot[R]{\scriptsize { \texttt{Trang {\thepage}/\pageref{LastPage}}}}
\fancyfoot[L]{\scriptsize {\texttt{Ứng dụng hệ thống nhúng nâng cao - 2017}}}
\renewcommand{\headrulewidth}{0.3pt}
\renewcommand{\footrulewidth}{0.3pt}
\definecolor{dkgreen}{rgb}{0,0.6,0}
\definecolor{gray}{rgb}{0.5,0.5,0.5}
\definecolor{mauve}{rgb}{0.58,0,0.82}
 
\lstset{frame=tb,
  language=R,
  aboveskip=3mm,
  belowskip=3mm,
  showstringspaces=false,
  columns=flexible,
  basicstyle={\small\ttfamily},
  numbers=none,
  numberstyle=\tiny\color{gray},
  keywordstyle=\color{blue},
  commentstyle=\color{dkgreen},
  stringstyle=\color{mauve},
  breaklines=true,
  breakatwhitespace=true,
  tabsize=3
}
\begin{document}
\begin{titlepage}
\begin{flushleft}
\begin{center}
\textbf{\Large ĐẠI HỌC QUỐC GIA THÀNH PHỐ HỒ CHÍ MINH}\\
\textbf{\Large TRƯỜNG ĐẠI HỌC BÁCH KHOA}\\
-----------------------$\bigstar$-----------------------
\end{center}
\end{flushleft}
\vspace{1cm}
\begin{figure}[h!]
\begin{center}
\includegraphics[width=45mm]{LogoBK.jpg}
\end{center}
\end{figure}
\vspace{1cm}
\begin{center}
\begin{tabular}{c}
\hline
\textbf{{\large BÁO CÁO TUẦN 17}}\\
\multicolumn{1}{l}{\textbf{{\LARGE LUẬN VĂN TỐT NGHIỆP}}}\\
~~\\
\hline
\end{tabular}

%\textbf{{\LARGE ĐỀ TÀI 4: BÀI TOÁN TIỀN LƯƠNG HƯU}}
\end{center}
\vspace{2cm}
\begin{minipage}[t]{0.650\linewidth}
GVGD: Phạm Hoàng Anh
\end{minipage}
\begin{minipage}[t]{0.90\linewidth}
Họ \& Tên:\\
\begin{tabular}{lcl}
Nguyễn Hương  &$\displaystyle{-}$& $\textsc{1411646}$ \\
Bùi Thanh Tùng  &$\displaystyle{-}$& $\textsc{1414517}$ \\
%\hline 

\end{tabular} 
\end{minipage}
\\ \\ \\ \\ \\ \\ \\

\vspace{3cm}
\begin{center}
\begin{center}
Tp.HCM, tháng 05 năm 2018
\end{center}
\end{center}
\end{titlepage}
\newpage
\begin{center}
{\color{blue}  {\tableofcontents}}
 \end{center} %summary insertion
\newpage
\section{Tất cả phần việc đã làm được tới hiện tại }
\textrm{- Điều khiển được các tư thế: Đi thẳng, sang phải, sang trái, quay trái, quay phải, cúi xuống bê đồ, đá chân}.\par
\textrm{- Phát được wifi edison từ board edison khi cấp nguồn ngoài một cách tự động}.\par
\textrm{- Hiện thực được website chứa giao diện nút nhấn để điều khiển robot qua wifi}.\par
\textrm{- Viết được app dùng API phát hiện giọng nói của google dùng để gửi lệnh điều khiển robot qua edison. Lệnh cơ bản : turn left, turn right, up, down}.\par
\textrm{- Kết nối, cấu hình và điều khiển thành công robot qua PS2. Đây sẽ là cách điều khiển chính khi demo}.\par
\textrm{- Khởi động và cấu hình BLE trên board edison. Dùng app BlueTerm có sẵn đã kết nối được với BLE edison}.\par
\textrm{- Đã gửi được tín hiệu từ app BlueTerm sang edison, và từ edison sang BlueTerm}.\par
\begin{center}
    \begin{figure}[htp]
    \begin{center}
	\includegraphics[scale=.5]{BLE_APP.jpg}
    \end{center}
    \caption{Gửi dữ liệu giữa App điện thoại và Edison}
    \label{refhinh1}
    \end{figure}
\end{center}
\textrm{- Đọc giá trị sensor trên 9DOF block. Xử lý được giá trị để phát hiện được té ngã.}.\par
\begin{center}
    \begin{figure}[htp]
    \begin{center}
     \includegraphics[scale=.4]{9DOF.png}
    \end{center}
    \caption{Thông số Accellometer đọc được và phát hiện ngã}
    \label{refhinh1}
    \end{figure}
\end{center}
\newpage
\newpage
\section{Khảo sát chuyển động khi cho robot di chuyển 10m }
\textrm{- Robot đã di chuyển được quãng đường 10m mà không phải can thiệp ngoại lực bên ngoài}.\par
\textrm{- Tuy nhiên robot đã đi một đường lệch không phải đường thẳng hoàn toàn}.\par
\textrm{- Vận tốc đo đạc được là: trung bình 2.5mét/1phút hoặc chạy hết 10 mét trong vòng 5 phút}.\par
\textrm{- Nhóm sẽ quay clip demo và gửi thầy nhanh nhất có thể.}.\par
\newpage
\newpage
\section{Các phần việc đang tiến hành }
\textrm{- Phát hiện hướng ngã, lực tác động lên robot để tự động cân bằng cho robot}.\par
\textrm{- Thiết kế giao diện app Bluetooth}.\par
\textrm{- Phát triển app Bluetooth riêng dựa trên app có sẵn}.\par
\newpage
\newpage
\section{Dự kiến báo cáo tiến độ tuần sau }
\textrm{- Clip đầy đủ về tất cả các phương pháp điều khiển đã làm được (BLE có thể có nếu hoàn chỉnh)}.\par
\textrm{- Trong clip có cho robot thực hiện đi thẳng một khoảng cách xa (Tạm thời 10m)}.\par
\textrm{- Báo cáo sơ bộ luận văn}.\par
\newpage
\end{document}